% \documentclass[10pt,twoside]{article}
% \usepackage[utf8]{inputenc}
\usepackage[a4paper,
scale={0.9,0.85},
bindingoffset=1cm
]{geometry}
% \usepackage[english,german,dutch]{babel}
\providecommand\doclangs{english,german,dutch}
\usepackage[\doclangs]{babel}
% \usepackage[dvipsnames]{xcolor}
\usepackage{rotating}
\usepackage{graphicx}
\usepackage[titletoc,toc,title]{appendix}
\usepackage{wrapfig}
\usepackage{tabularx}
\usepackage{caption}
\usepackage{subcaption}
\usepackage{textcomp}
\usepackage{listings}
\usepackage{times}
\usepackage{amsmath}
\usepackage{multicol}
\usepackage{colortbl}
\usepackage{times}
\usepackage{fancybox}
\usepackage{mdframed}
\usepackage{afterpage}
\usepackage{enumitem}
\usepackage{mdwlist}
\usepackage{ifthen}
\usepackage{lineno}
\usepackage{eurosym}
\usepackage[mackeys=symbols]{menukeys}
%% to be able to simply type the (utf-8) euro sign
\DeclareUnicodeCharacter{20AC}{\euro}
%% 
\linenumbers
\pagewiselinenumbers
\modulolinenumbers[3]
\addtolength\voffset{4mm}
\setlength\parindent{0pt}
\setlength\parskip{.25\baselineskip}
\usepackage{setspace}
\usepackage[code=Code39,X=.5mm,ratio=2.25,H=15mm]{makebarcode}
\providecommand\Spacing{}
\Spacing
\InputIfFileExists{../defaults.tex}{}{}
\InputIfFileExists{date.tex}{}{}
\InputIfFileExists{packages.tex}{}{}
\providecommand\StickNr{EXAM000}
\providecommand\StudentNumber{1234567}
\providecommand\StudentId{Jan~Klaassen~1234567}
\providecommand\Author{Pietje~Puk}
\providecommand\Lang{\StickNr}
\providecommand\ExamTitle{Prutsen~op~Niveau}
\providecommand\ExamDate{\today}
%% to be able to simply type the (utf-8) euro sign
\DeclareUnicodeCharacter{20AC}{\euro}
%% headers
\usepackage{fancyhdr}
\renewcommand\footrulewidth{.2pt}
\fancyhf{}
\lfoot[\ExamTitle]{\ExamTitle}
\rfoot[\ExamDate]{\ExamDate}
\rhead[\textbf{\Lang}]{\textbf{\Lang}}
\lhead[\large\bf\StudentId]{\large\bf\StudentId}
\cfoot[\thepage]{\thepage}
\fancypagestyle{plain}{
  \lfoot[\ExamTitle]{\ExamTitle}
  \rfoot[\ExamDate]{\ExamDate}
  \rhead[\textbf{\Lang}]{\textbf{\Lang}}
  \lhead[\large\bf\StudentId]{\large\bf\StudentId}
  \cfoot[\thepage]{\thepage}
}
\pagestyle{fancy}
\addtolength\headheight{0.2\baselineskip}
\newcommand\Code[1]{\texttt{\textbf{\color[rgb]{.3,0,.3}#1}}} %
\usepackage{fontystasks}
\def\CommitReminder{
  \vspace{.5\baselineskip}
  \hrule
  \vspace{\baselineskip}
}
% \begin{center}
%   \fbox{\Large\textbf{\sf{}** COMMIT REMINDER **}}
% \end{center}}

\def\Closing{
  \vfill
  \EN{\typeout{EN Closing called}}
  \begin{center}
    \shadowbox{\begin{minipage}{.8\textwidth}
        \begin{center}\Large\bf\sffamily\color{Mahogany}
          % \EN{End of the assessment\\Commit for the last time before
          % you close down\\Have this commit check by the supervision.}
          % \NL{Einde van het assessment.\\Voer een laatste commit uit
          % voordat je afsluit!\\Laat deze commit door het toezicht controleren}
          % \DE{Ende des Assessments.\\Führen Sie ein letztes Commit aus
          % bevor Sie abschließen!\\Lassen Sie diesen commit vom
          % Aufsicht überprüfen.}
          
          \EN{When you are done, close the IDE and shut down your computer.}
          \NL{Wanneer je klaar bent dan sluit je de IDE af en sluit je
            de computer af.}
          \DE{Wenn Sie fertig sind , schließen Sie  die IDE und fahren
            Sie den Computer runter.}
          
          \EN{Hand in the USB stick and all received papers.}
          \NL{Lever de USB-stick en alle uitgereikte papieren in.}
          \DE{Reichen Sie den USB-Stick mit allen Papieren ein.}
        \end{center}
      \end{minipage}}
  \end{center}
  \vfill
}

\def\TDDNote{%
  \begin{center}\sf
    {\begin{center}\Large\textbf{Work and commit Test Driven}\end{center}}
    \framebox{
      \begin{minipage}{.95\linewidth}
        In this exam you are required to work test driven. This makes
        the use of the repository on your exam stick mandatory. 
        \begin{itemize*}
        \item After
          every implementation of a test, and the test being red, do a
          commit.
        \item Then implement the business method. When the test now
          is green, commit.
        \item Every time you add a test aspect (modify,
          improve), making the test red again, do a commit.
        \item Then improve the implementation.
        \end{itemize*}
        
        {\large \textbf{Add useful commit messages like \texttt{'added
              test TASK\_3B'}}} 
        
        \vspace{.5\baselineskip}
        We look a the commit history and browse through your
        commits. Working in any other way will cost you (SEN1)
        points. Working test  driven will also force you to study the
        exact specification, signature and all, of your methods, which
        typically are fixed and should not be changed, and thereby
        improving your chances for JAVA2 as well.
        Working with subversion on the command line is remarkably easy:
        \begin{itemize*}
        \item Op the terminal in the examproject folder. (open the exam
          folder by clicking on it, then right click inside the exam
          folder and open a terminal.)
        \item In the terminal type:\\
        \end{itemize*}\vspace{-\baselineskip}
        \hspace{4em}\framebox{
          \begin{minipage}{.5\linewidth}\ttfamily\bfseries
            svn up\\
            svn ci -m'completed TASK\_1A'
          \end{minipage}
        }
        \begin{itemize*}
        \item Do not forget the quotes.
        \item These commands will commit your work to the repository.
        \item You can leave the terminal open, you will do more commits.
        \end{itemize*}
      \end{minipage}
    }
  \end{center}

}

\definecolor{lbcolor}{rgb}{0.9,0.9,0.9}
\lstset{
  backgroundcolor=\color{lbcolor},
  tabsize=4,
  %	rulecolor=,
  language=java,
  morekeywords={enum},
  basicstyle={\normalsize\bfseries\ttfamily},
  upquote=true,
  % aboveskip={1.5\baselineskip},
  columns=fixed,
  showstringspaces=false,
  extendedchars=true,
  breaklines=true,
  prebreak = \raisebox{0ex}[0ex][0ex]{\ensuremath{\hookleftarrow}},
  frame=single,
  showtabs=false,
  showspaces=false,
  showstringspaces=false,
  identifierstyle=\ttfamily,
  keywordstyle=\color[rgb]{0,0,1},
  commentstyle=\color[rgb]{0,0.545,0},
  stringstyle=\color[rgb]{0.627,0.126,0.941},
  numberstyle={\tiny\sffamily}
}
\lstdefinestyle{block}{
  basicstyle={\scriptsize}
}
\lstMakeShortInline|
%% listing captions
\DeclareCaptionFont{white}{ \color{white} }
\DeclareCaptionFormat{listing}{
  \hspace{-2.5em}\colorbox[cmyk]{0.43, 0.35, 0.35,0.01 }{
    \parbox{\linewidth}{\hspace{10pt}#1#2#3}
  }
}
\captionsetup[lstlisting]{ format=listing, labelfont=white,
  textfont=white, singlelinecheck=false, margin=0pt,
  font={sf,bf,footnotesize} }
\usepackage{hyperref}
\providecommand\NL[1]{}
\providecommand\DE[1]{}
\providecommand\EN[1]{}
\title{\ExamTitle}
\date{\ExamDate}
\author{\Author}
%%% Local Variables: 
%%% mode: plain-tex
%%% TeX-master: t
%%% End: 
