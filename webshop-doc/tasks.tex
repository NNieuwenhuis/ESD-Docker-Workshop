\section{Your practical tasks}
You can find the SEN1 tasks in the \Code{webshopModel} project.
There are no SEN1 tasks in the webshop web project.

In essence you should test and develop (yes in that order) 
methods in the ProductContainer class. The tests and methods
you have to program are marked as such in the source code.

It is best to follow the given task numbering scheme, for instance task
T01\_A (the test) should be followed by T01\_B (the implementation
to that test).

The projects are available in a repository, which resides on the USB
stick as well. The projects have been checked out into a sandbox called
\texttt{examproject-EXAMxyz} on your desktop. NetBeans will do the right thing
if you \Code{svn commit} or \Code{svn checkout}. In this way you can
use subversion as a safety net and also keep the good way of working
of TDD: RED, GREEN, REFACTOR. 

\begin{center}
  {\bf\sffamily Important hint: Work Test Driven}
\end{center}

\begin{description*}
\item [Develop a test:] (Test one aspect of the method that must be
  implemented). When the test compiles but fails (red) you have a
  working test. Now commit with  log-comment \texttt{test {\bfseries\color{BrickRed}red}}.
  
\item[Implement an aspect:] Now implement the aspect as in ``make the test
  pass''. When it is green, again commit with  log-comment
  \texttt{test {\bfseries\color{OliveGreen}green}}.
\item[Repeat:] The above until all aspects have tests and are implemented
  and your world is green.
\end{description*}

\begin{center}
  {\large\sffamily{}Commit after \textit{every successful run},  \textbf{Red} or
    \textbf{Green}. A failing (red) test is a good test!\\Getting the test green
    afterwards is only better if it was red before.}
\end{center}

\textbf{Last remarks:}
\begin{itemize*}
\item Your development environment provides code coverage by means of the
\textbf{tikione/jacoco} NetBeans plug-in. This can create html
coverage reports which shows you what code you missed, test wise.
\item If a test annoys you, you can temporarily switch it off by adding the
\lstinline{@Ignore} annotation to that test or even to the test class.

% All tasks are given as TODO's in the code.
% Some requirements are best described by referring to the
% opening figure of this document~\ref{fig:screenshot} on
% page~\pageref{fig:screenshot}.

% The left hand side is the shop, top is inventory, bottom is cart.
\item There is NO \Code{JSF-xhtml}\ work in this exam. You could of course use the
application (start webshop project in glassfish).
% In the JSF pages you have to make sure that:
% \begin{itemize*}
% \item Only render an add or remove button, when it makes sense of the
%   user to press it. In particular:
%   \begin{itemize*}
%   \item The \textbf{Add to cart} button should only be rendered when the product
%     is 1. not yet in the car AND 2. the product is available in the inventory.
%   \item The text ``Not on stock'' should be rendered instead if the
%     product is sold out (qty in  inventory equals 0).
%   \item The text ``already in cart'' should be rendered when product
%     already in cart.
%   \item In the cart, the plus button on items should only be rendered
%     when there still is stock of said item.
%   \end{itemize*}
% \item On the invoice only render the SPECIAL PRICE and ``YOUR
%   ADVANTAGE WITH BONUS'' when the a special price code is activated.
% \item The BONUS code button (Activate) should result in activating
%   the discount code. The way the price advantage (for the customer) is
%   calculated depends on the special price calculator activated by the
%   given code. See listing~\ref{lst:bonus}
% \end{itemize*}

% \begin{lstlisting}[language=Java,caption={\label{lst:bonus}bonus codes are defined in the source code},basicstyle={\scriptsize}]
%   public static final String BONUS = "BONUS";
%   public static final String VAT_FREE_FILM = "NO VAT ON FILMS";
%   public static final String DOUBLE_CRIPS = "CRISPS";
% \end{lstlisting}
\item Bonus codes understood by the application are: ``BONUS'',
  ``NO~VAT~ON~FILMS'' and ``CRISPS''.
\item You can find the tasks and documentation in the source code of both projects.
Because the documentation uses Javadoc features, so before you start
coding, you should generate the java-doc from the
\Code{webshopModel2015} project.
\item The tasks (\textbf{TODO}'s) can be found in NetBeans IDE by pressing 
\keys{CRTL+6}, or on a German keyboard \keys{STRG+6}
\item The tasks are numbered. We tried to make these numbers guide you
through the exam. Most tasks have and A and a B part. In such cases,
the A part counts for testing (SEN1), the B part for programming
(PRO2, practical part).

\item The first task (create a ``refreshment'') should resolve the issues
the compiler has with the initial state of the webshopModel project.
\end{itemize*}
% \clearpage
\subsection{Exam tasks}
The tasks list in this exam, collected from the source code is listed
below.
\nolinenumbers
\sloppypar
\setlength{\columnseprule}{.1pt}
\begin{multicols}{2}
  \small
\begin{description}
  \item[T01\_A1] Type of Warsteiner test, asserts
  \item[T01\_A2] Create predicate to check if cart correctly checks for Booze
  \item[T01\_B1] implement Booze class.
  \item[T01\_B2] implement ageOk
  \item[T02\_A] set up mock cart with return values.
  \item[T02\_B] lambda for two for one price
  \item[T03\_A] testSaveAndLoad of im persistence
  \item[T03\_B] implement IMInvoiceMapper.save(Invoice). read javadoc first
  \item[T04\_A] testGetTotalPriceExcludingVAT
  \item[T04\_B] implement getTotalPriceExcludingVat using for-loop
  \item[T05\_A] reduction on Film
  \item[T05\_B] compute reduction.
  \item[T06\_A] test mapCartToLines
  \item[T06\_B] impl Invoice.mapToCartLines
  \item[T07\_A] test save and load
  \item[T07\_B] implement part of PGDBInvoiceMapper.load()
\end{description}

\end{multicols}

